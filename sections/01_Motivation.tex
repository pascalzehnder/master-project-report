\section{Motivation}
Students in the field of Banking and Finance at the University of Zurich learn a lot about asset management as well as business management aspects during various courses. In order to give their students an opportunity to apply their knowledge and understanding of the portfolio management process, a Portfolio Management Game is used within a seminar for Master students. According to the Department of Banking and Finance, the main targets of the simulation are the following learning outcomes:
\begin{itemize}
  \item Students learn how money can be systematically invested in financial markets.
  \item Students learn what factors make financial market forecasts possible and what limitations the models have for forecasting.
  \item Students learn which factors are relevant for the success (performance) of investments and can distinguish between factors that promise short-term success and aspects that are relevant in the long term.
\end{itemize}

In the simulation, the students act in groups as heads of a bank’s portfolio management and are responsible for both the investment strategy and the business management of investment funds. The games content focuses on a structured investment process, which covers the steps from getting to know different customer types of the bank, selecting a suitable long term Strategic Asset Allocation, adjust it according to the status of the economy for the Tactical Asset Allocation and select appropriate titles for the depot realization. A competition for better performance for their clients among the different student teams makes the learning process entertaining.

However, the currently used version of the Portfolio Management Game is technically and didactically outdated. The currently used Portfolio Management Game (Portfoliomanagement SIM) was initially developed in 2005 by the Department of Banking and Finance at the University of Zurich in cooperation with the Swiss bank Julius Bär as well as game solution ag. A rapid technological development since that time allows new perspectives and possibilities in the field of game-based learning.

This project aims to redesign and reprogram the existing Portfolio Management Game so that it is up to date both from a content, technical and didactic point of view. Therefore, a project team from the Department of Banking and Finance and Department of Informatics have decided to start a cooperation in order to elaborate a modern version of the Portfolio Management Game. From the content side, a project team around Prof. Dr. Markus Leippold and Dr. Benjamin Wilding have been assembled. From the IT side, Prof. Dr. Chat Wacharamanotham together with two students (also the authors of this work) are in the lead. Both student members of the project team work at the Department of Banking and Finance UZH as web developers parallel to their studies achieving their Master’s degree in Informatics. Both interested in developing applications from scratch and analyzing the procedure of financial processes. By redeveloping the application the Department of Banking and Finance wants to achieve a sustainable simulation of a typical portfolio management process. The simulation should help the students within their learning process by focussing on practical decision making, building up on their theoretical knowledge. 

The present work focuses more on the IT side of the project. Subsequently, a more detailed description of the project scope is provided.
