\section{Project Overview}
\label{sec:project_overview}

The focus of this project lies in the development of an improved version of the old ''Portfolio management SIM''. The old version, having been created in 2005, has several critical pain points that should be accounted for in the new project. To provide a general overview of the areas in need of improvements, we first shortly provide an overview of these problems. The following sections then focus on the project overview and procedures as well as the main goals and project scope.

\subsection{Major Issues in the Old Version}
As previously mentioned, the old version of the game needs to be revised. According to the involved members of the DBF, the development of a new simulation game intends to enhance different existing elements while also introducing new features:

\begin{itemize}
  \item Specially configured hardware was necessary to play a game with the old application, as it required the installation of a native Windows application without networking capabilities. After each round of a game, the supervisors had to export the decisions of the students onto memory sticks. The contents of these sticks then had to be transferred to a central device with administrative access (on the same windows native application with game master access) in order to calculate the results for a period. Especially due to this reason, only a limited amount of teams could play the game simultaneously. Through the transformation into a web-based environment, the simulation can be used independently of time and location, whereby the number of participants can be scaled to higher numbers. It is therefore also conceivable that the simulation could be offered not only to students at the Department of Business, Economics, and Informatics but also to students of other faculties and in larger environments.
  \item Due to the use of real historical financial market data in the previous simulation, students can improve their success by researching past share prices. This means that it is not necessarily the students who invest the money in a scientifically meaningful way who score best, but those who carry out the best research. A simulation shall enable the use of randomized data simulated by mathematical processes before the execution of a period. On the one hand, this would improve fairness in the simulation. On the other hand, it would also show that it is not possible to forecast financial market data precisely. This simulation is not directly part of this project but it should be possible to integrate with a future simulation without major issues.
  \item Students should be shown that the forecasting ability on financial markets is limited and that the investment of funds should, therefore, be based on fewer, theoretically sound principles. With  simulated data, it would be clear that misconduct can lead to short-term profits, but that systematic behavior is decisive for success in the long term (across multiple independently randomized periods)
\end{itemize}


\newpage
\subsection{Project Focus and Goals}
The main focus as summarized by the DBF team should lie on the following items and characteristics:
\begin{itemize}
  \item \textbf{Usability:} The different game sequences within the market model should be well-designed and comprehensible for the game master and the players.
  \item \textbf{Scalability:} Design a well functioning game that can be played with a smaller and larger number of students (potentially even scaled up to an assessment level course).
  \item \textbf{Modularity:} Basic and an advanced version(s) and settings allow the game to be used on different levels of study.
  \item \textbf{Key Characteristics}
  \begin{itemize}
    \item The game should be self-explanatory and intuitive.
    \item A brief and concise game documentation is to be created (part of the project report).
    \item The game contains at least the useful components of the old game.
    \item Informative innovative graphical output (as self-explanatory as possible) for the instructor (based on the current Excel-evaluation and further ideas we provide in advance) and students (see ``Teilnehmerbericht'' of the old game and a depiction of performance-attribution).
    \item New market model created by the Department of Banking and Finance in an Excel spreadsheet which will be the base for the implementation of the authors.
  \end{itemize}
\end{itemize}



\subsection{Project Procedure}
The pfm-game project was initiated and originally planned by the project management team of the DBF. The specifications for the implementation were defined in cooperation with the developers. Furthermore, a student from the DBF was given the task to prepare an economic model for the data simulation in form of a master thesis, with the goal of being able to integrate the model with the pfm-game project later on.\\

In the next phase, the architecture was conceptualized according to the main project goals and plans for requirements engineering were created according to best practice procedures as discussed with the supervisor from IFI. During the requirements engineering phase, user stories and initial mockups were created in cooperation with the DBF team, incorporating collected knowledge from user observations, interviews, as well as other means of data collection (further described in \Cref{sec:methodology}).\\

The main development phase was then started with an initial architecture implementation as well as a focus on continuous integration and automated testing for critical API resources. An initial prototype of the application was developed with mock asset data, as real data or market modeling were not yet available or defined by the DBF team. Once a market model was available in an Excel format, the focus of the project was set on implementing the model as a Python service. The model was then integrated with the existing application prototype and first efforts were made to integrate the entire application with real historical data. \\

The integration achieved during the course of this project is based on historical stock prices. Due to open issues in the simulation project that still remain, an integration with simulated data or industry sector portfolios was not yet possible. However, this integration is planned for the near future. Towards the end of the project, an initial model validation and application testing phase followed. During initial validation of the model, some issues that require further definition and thoughts by the DBF team were already uncovered and will be amended in future releases. \\


